\documentclass[a4paper,12pt,reqno,superscriptaddress,nofootinbib]{article}
%\documentclass{article}
\usepackage[centertags]{amsmath}
\usepackage{amsfonts}
\usepackage{amssymb}
\usepackage{amsthm}
\usepackage{newlfont}
\usepackage{stmaryrd}
\usepackage{mathrsfs}
\usepackage{mathtools}
\usepackage{euscript}
\usepackage{graphicx}
\usepackage{enumerate}
\usepackage{color}
\usepackage[margin=1in]{geometry}


\usepackage{hyperref}
\usepackage{physics}
\usepackage{subcaption}
\usepackage{siunitx}


% THEOREM-LIKE ENVIRONMENTS -----------------------------------------

\theoremstyle{plain}
\newtheorem{theorem}{Theorem}[section]
\newtheorem{corollary}[theorem]{Corollary}
\newtheorem{proposition}[theorem]{Proposition}
\newtheorem{lemma}[theorem]{Lemma}
\theoremstyle{definition}
\newtheorem{definition}[theorem]{Definition}
\newtheorem{assumption}[theorem]{Assumption}
\newtheorem{condition}[theorem]{Condition}
\newtheorem{conjecture}[theorem]{Conjecture}

\theoremstyle{remark}
\newtheorem{remark}[theorem]{Remark}
\newtheorem{note}[theorem]{Note}
\newtheorem{notes}[theorem]{Notes}
\newtheorem{example}[theorem]{Example}



% \MATHOPERATOR -----------------------------------------------------

\DeclareMathOperator{\ex}{Ex} \DeclareMathOperator{\ext}{Ext}
\DeclareMathOperator{\Int}{Int} \DeclareMathOperator{\supp}{Supp}
\DeclareMathOperator{\const}{Const}
\newcommand{\re}{\Re\,}
\newcommand{\im}{\Im\,}

\newcommand{\HH}{\mathsf{H}}
\newcommand{\PP}{\mathsf{P}}
\newcommand{\RR}{\mathsf{R}}

% GREEK - 2 letters ------------------------------------------------

\let\al=\alpha %\let\be=\beta
\let\de=\delta \let\ep=\epsilon
\let\ve=\varepsilon \let\vp=\varphi \let\ga=\gamma \let\io=\iota
\let\ka=\kappa \let\la=\lambda \let\om=\omega \let\vr=\varrho
\let\si=\sigma \let\vs=\varsigma \let\th=\theta \let\vt=\vartheta
\let\ze=\zeta \let\up=\upsilon

\let\De=\Delta \let\Ga=\chi \let\La=\Lambda \let\Om=\Omega
\let\Th=\Theta \let\Up=\Upsilon

% \MATHCAL - \ca ----------------------------------------------------


\newcommand{\be}{\begin{equation}}
	\newcommand{\en}{\end{equation}}
\def\fr{\frac}
\def\th{\theta}
\def\al{\alpha}
\def\om{\omega}
\def\ka{\kappa}
\def\ii{\textrm i}
\def\ee{\textrm e}
\def\e{\textrm e}
\def\vp{\varphi}
\def\sgn{{\textrm{sgn}}}
\def\ud{\textrm{d}}
\def\bn{{\boldsymbol \nabla}}
\def\setR{\mathbb{R}}

\let\lam=\lambda
\let\si=\sigma \let\vs=\varsigma \let\th=\theta \let\vt=\vartheta
\let\ze=\zeta \let\up=\upsilon

\let\De=\Delta \let\Ga=\Gamma \let\La=\Lambda \let\Om=\Omega
\let\Th=\Theta \let\Up=\Upsilon
\let\om=\omega



\newcommand{\caA}{{\mathcal A}}
\newcommand{\caB}{{\mathcal B}}
\newcommand{\caC}{{\mathcal C}}
\newcommand{\caD}{{\mathcal D}}
\newcommand{\caE}{{\mathcal E}}
\newcommand{\caF}{{\mathcal F}}
\newcommand{\caG}{{\mathcal G}}
\newcommand{\caH}{{\mathcal H}}
\newcommand{\caI}{{\mathcal I}}
\newcommand{\caJ}{{\mathcal J}}
\newcommand{\caK}{{\mathcal K}}
\newcommand{\caL}{{\mathcal L}}
\newcommand{\caM}{{\mathcal M}}
\newcommand{\caN}{{\mathcal N}}
\newcommand{\caO}{{\mathcal O}}
\newcommand{\caP}{{\mathcal P}}
\newcommand{\caQ}{{\mathcal Q}}
\newcommand{\caR}{{\mathcal R}}
\newcommand{\caS}{{\mathcal S}}
\newcommand{\caT}{{\mathcal T}}
\newcommand{\caU}{{\mathcal U}}
\newcommand{\caV}{{\mathcal V}}
\newcommand{\caW}{{\mathcal W}}
\newcommand{\caX}{{\mathcal X}}
\newcommand{\caY}{{\mathcal Y}}
\newcommand{\caZ}{{\mathcal Z}}

\newcommand{\bR}{\mathbb{R}}

% \MATHBB - \bb -----------------------------------------------------

\newcommand{\bba}{{\mathbb a}}
\newcommand{\bbb}{{\mathbb b}}
\newcommand{\bbc}{{\mathbb c}}
\newcommand{\bbd}{{\mathbb d}}
\newcommand{\bbe}{{\mathbb e}}
\newcommand{\bbf}{{\mathbb f}}
\newcommand{\bbg}{{\mathbb g}}
\newcommand{\bbh}{{\mathbb h}}
\newcommand{\bbi}{{\mathbb i}}
\newcommand{\bbj}{{\mathbb j}}
\newcommand{\bbk}{{\mathbb k}}
\newcommand{\bbl}{{\mathbb l}}
\newcommand{\bbm}{{\mathbb m}}
\newcommand{\bbn}{{\mathbb n}}
\newcommand{\bbo}{{\mathbb o}}
\newcommand{\bbp}{{\mathbb p}}
\newcommand{\bbq}{{\mathbb q}}
\newcommand{\bbr}{{\mathbb r}}
\newcommand{\bbs}{{\mathbb s}}
\newcommand{\bbt}{{\mathbb t}}
\newcommand{\bbu}{{\mathbb u}}
\newcommand{\bbv}{{\mathbb v}}
\newcommand{\bbw}{{\mathbb w}}
\newcommand{\bbx}{{\mathbb x}}
\newcommand{\bby}{{\mathbb y}}
\newcommand{\bbz}{{\mathbb z}}

\newcommand{\bbA}{{\mathbb A}}
\newcommand{\bbB}{{\mathbb B}}
\newcommand{\bbC}{{\mathbb C}}
\newcommand{\bbD}{{\mathbb D}}
\newcommand{\bbE}{{\mathbb E}}
\newcommand{\bbF}{{\mathbb F}}
\newcommand{\bbG}{{\mathbb G}}
\newcommand{\bbH}{{\mathbb H}}
\newcommand{\bbI}{{\mathbb I}}
\newcommand{\bbJ}{{\mathbb J}}
\newcommand{\bbK}{{\mathbb K}}
\newcommand{\bbL}{{\mathbb L}}
\newcommand{\bbM}{{\mathbb M}}
\newcommand{\bbN}{{\mathbb N}}
\newcommand{\bbO}{{\mathbb O}}
\newcommand{\bbP}{{\mathbb P}}
\newcommand{\bbQ}{{\mathbb Q}}
\newcommand{\bbR}{{\mathbb R}}
\newcommand{\bbS}{{\mathbb S}}
\newcommand{\bbT}{{\mathbb T}}
\newcommand{\bbU}{{\mathbb U}}
\newcommand{\bbV}{{\mathbb V}}
\newcommand{\bbW}{{\mathbb W}}
\newcommand{\bbX}{{\mathbb X}}
\newcommand{\bbY}{{\mathbb Y}}
\newcommand{\bbZ}{{\mathbb Z}}

\newcommand{\opunit}{\text{1}\kern-0.22em\text{l}}
\newcommand{\funit}{\mathbf{1}}

% \MATHFRAK - \fr ---------------------------------------------------

\newcommand{\ong}{\backsimeq}
\newcommand{\Z}{\mathbb Z}
\newcommand{\R}{\mathbb R}
\newcommand{\CC}{\mathbb C}
%\newcommand{\de}{\text{d}}
\newcommand{\dt}{\text{d}t}
\newcommand{\dx}{\text{d}x}
\newcommand{\dV}{\text{d}V}
\newcommand{\dr}{\text{d}r}
\newcommand{\ds}{\text{d}s}
\newcommand{\BE}{\text{BE}}
\newcommand{\een}{\mathds{1}}
\newcommand{\id}{\textrm{d}}



\newcommand{\fra}{{\mathfrak a}}
\newcommand{\frb}{{\mathfrak b}}
\newcommand{\frc}{{\mathfrak c}}
\newcommand{\frd}{{\mathfrak d}}
\newcommand{\fre}{{\mathfrak e}}
\newcommand{\frf}{{\mathfrak f}}
\newcommand{\frg}{{\mathfrak g}}
\newcommand{\frh}{{\mathfrak h}}
\newcommand{\fri}{{\mathfrak i}}
\newcommand{\frj}{{\mathfrak j}}
\newcommand{\frk}{{\mathfrak k}}
\newcommand{\frl}{{\mathfrak l}}
\newcommand{\frm}{{\mathfrak m}}
\newcommand{\frn}{{\mathfrak n}}
\newcommand{\fro}{{\mathfrak o}}
\newcommand{\frp}{{\mathfrak p}}
\newcommand{\frq}{{\mathfrak q}}
\newcommand{\frr}{{\mathfrak r}}
\newcommand{\frs}{{\mathfrak s}}
\newcommand{\frt}{{\mathfrak t}}
\newcommand{\fru}{{\mathfrak u}}
\newcommand{\frv}{{\mathfrak v}}
\newcommand{\frw}{{\mathfrak w}}
\newcommand{\frx}{{\mathfrak x}}
\newcommand{\fry}{{\mathfrak y}}
\newcommand{\frz}{{\mathfrak z}}

\newcommand{\frA}{{\mathfrak A}}
\newcommand{\frB}{{\mathfrak B}}
\newcommand{\frC}{{\mathfrak C}}
\newcommand{\frD}{{\mathfrak D}}
\newcommand{\frE}{{\mathfrak E}}
\newcommand{\frF}{{\mathfrak F}}
\newcommand{\frG}{{\mathfrak G}}
\newcommand{\frH}{{\mathfrak H}}
\newcommand{\frI}{{\mathfrak I}}
\newcommand{\frJ}{{\mathfrak J}}
\newcommand{\frK}{{\mathfrak K}}
\newcommand{\frL}{{\mathfrak L}}
\newcommand{\frM}{{\mathfrak M}}
\newcommand{\frN}{{\mathfrak N}}
\newcommand{\frO}{{\mathfrak O}}
\newcommand{\frP}{{\mathfrak P}}
\newcommand{\frQ}{{\mathfrak Q}}
\newcommand{\frR}{{\mathfrak R}}
\newcommand{\frS}{{\mathfrak S}}
\newcommand{\frT}{{\mathfrak T}}
\newcommand{\frU}{{\mathfrak U}}
\newcommand{\frV}{{\mathfrak V}}
\newcommand{\frW}{{\mathfrak W}}
\newcommand{\frX}{{\mathfrak X}}
\newcommand{\frY}{{\mathfrak Y}}
\newcommand{\frZ}{{\mathfrak Z}}

% \BOLDSYMBOL - \bs -------------------------------------------------

\newcommand{\bsa}{{\boldsymbol a}}
\newcommand{\bsb}{{\boldsymbol b}}
\newcommand{\bsc}{{\boldsymbol c}}
\newcommand{\bsd}{{\boldsymbol d}}
\newcommand{\bse}{{\boldsymbol e}}
\newcommand{\bsf}{{\boldsymbol f}}
\newcommand{\bsg}{{\boldsymbol g}}
\newcommand{\bsh}{{\boldsymbol h}}
\newcommand{\bsi}{{\boldsymbol i}}
\newcommand{\bsj}{{\boldsymbol j}}
\newcommand{\bsk}{{\boldsymbol k}}
\newcommand{\bsl}{{\boldsymbol l}}
\newcommand{\bsm}{{\boldsymbol m}}
\newcommand{\bsn}{{\boldsymbol n}}
\newcommand{\bso}{{\boldsymbol o}}
\newcommand{\bsp}{{\boldsymbol p}}
\newcommand{\bsq}{{\boldsymbol q}}
\newcommand{\bsr}{{\boldsymbol r}}
\newcommand{\bss}{{\boldsymbol s}}
\newcommand{\bst}{{\boldsymbol t}}
\newcommand{\bsu}{{\boldsymbol u}}
\newcommand{\bsv}{{\boldsymbol v}}
\newcommand{\bsw}{{\boldsymbol w}}
\newcommand{\bsx}{{\boldsymbol x}}
\newcommand{\bsy}{{\boldsymbol y}}
\newcommand{\bsz}{{\boldsymbol z}}

\newcommand{\bsA}{{\boldsymbol A}}
\newcommand{\bsB}{{\boldsymbol B}}
\newcommand{\bsC}{{\boldsymbol C}}
\newcommand{\bsD}{{\boldsymbol D}}
\newcommand{\bsE}{{\boldsymbol E}}
\newcommand{\bsF}{{\boldsymbol F}}
\newcommand{\bsG}{{\boldsymbol G}}
\newcommand{\bsH}{{\boldsymbol H}}
\newcommand{\bsI}{{\boldsymbol I}}
\newcommand{\bsJ}{{\boldsymbol J}}
\newcommand{\bsK}{{\boldsymbol K}}
\newcommand{\bsL}{{\boldsymbol L}}
\newcommand{\bsM}{{\boldsymbol M}}
\newcommand{\bsN}{{\boldsymbol N}}
\newcommand{\bsO}{{\boldsymbol O}}
\newcommand{\bsP}{{\boldsymbol P}}
\newcommand{\bsQ}{{\boldsymbol Q}}
\newcommand{\bsR}{{\boldsymbol R}}
\newcommand{\bsS}{{\boldsymbol S}}
\newcommand{\bsT}{{\boldsymbol T}}
\newcommand{\bsU}{{\boldsymbol U}}
\newcommand{\bsV}{{\boldsymbol V}}
\newcommand{\bsW}{{\boldsymbol W}}
\newcommand{\bsX}{{\boldsymbol X}}
\newcommand{\bsY}{{\boldsymbol Y}}
\newcommand{\bsZ}{{\boldsymbol Z}}

\newcommand{\scS}{{\mathscr S}}
\newcommand{\scD}{{\mathscr D}}


\newcommand{\bsalpha}{{\boldsymbol \alpha}}
\newcommand{\bsbeta}{{\boldsymbol \beta}}
\newcommand{\bsgamma}{{\boldsymbol \gamma}}
\newcommand{\bsdelta}{{\boldsymbol \delta}}
\newcommand{\bsepsilon}{{\boldsymbol \epsilon}}
\newcommand{\bsmu}{{\boldsymbol \mu}}
\newcommand{\bsomega}{{\boldsymbol \omega}}

\DeclareMathAlphabet{\mathpzc}{OT1}{pzc}{m}{it}
\newcommand{\pzo}{\mathpzc{o}}
\newcommand{\pzO}{\mathpzc{O}}


% ABBREVIATION ------------------------------------------------------

\newcommand{\fig}{Fig.\;}
\newcommand{\cf}{cf.\;}
\newcommand{\eg}{e.g.\;}
\newcommand{\ie}{i.e.\;}

% MISCELLANEOUS -----------------------------------------------------

\newcommand{\un}[1]{\underline{#1}}
\newcommand{\defin}{\stackrel{\text{def}}{=}}
\newcommand{\bound}{\partial}
\newcommand{\sbound}{\hat{\partial}}
\newcommand{\rel}{\,|\,}
\newcommand{\pnt}{\rightsquigarrow}
\newcommand{\pa}{_\bullet}
\newcommand{\nb}[1]{\marginpar{\tiny {#1}}}
\newcommand{\pair}[1]{\langle{#1}\rangle}
\newcommand{\0}{^{(0)}}
\newcommand{\1}{^{(1)}}
\newcommand{\2}{^{(2)}}
\newcommand{\tot}{_{\text{TOT}}}
\newcommand{\out}{_{\text{OUT}}}
\newcommand{\con}{_{\text{con}}}
%\newcommand{\id}{\textrm{d}}
\newcommand{\can}{\text{can}}
\newcommand{\tomean}{\stackrel{1}{\to}}
\newcommand{\rev}{_{\text{REV}}}
\newcommand{\irr}{_{\text{IRR}}}

\DeclareMathOperator{\cnst}{const}
\def\Z{\mathcal Z}
\def\cL{\mathscr L}
\def\cH{\mathscr H}
\def\cA{\mathcal A}
\def\R{\mathbb R}
\def\S{\mathcal S}
\def\mbf{\mathbf }

\def\dbar{{\mathchar'26\mkern-12mu d}}

\long\def\red#1{{\color{red}#1}}
\renewcommand\div{\mathop\mathrm{div}}


% New definition of square root:
% it renames \sqrt as \oldsqrt
\let\oldsqrt\sqrt
% it defines the new \sqrt in terms of the old one
\def\sqrt{\mathpalette\DHLhksqrt}
\def\DHLhksqrt#1#2{%
	\setbox0=\hbox{$#1\oldsqrt{#2\,}$}\dimen0=\ht0
	\advance\dimen0-0.2\ht0
	\setbox2=\hbox{\vrule height\ht0 depth -\dimen0}%
	{\box0\lower0.4pt\box2}}

\let\a=\alpha %\let\be=\beta
\let\de=\delta \let\ep=\epsilon
\let\ve=\varepsilon \let\vp=\varphi \let\ga=\gamma \let\io=\iota
\let\ka=\kappa  \let\om=\omega \let\vr=\varrho
\let\si=\sigma \let\vs=\varsigma \let\th=\theta \let\vt=\vartheta
\let\ze=\zeta \let\up=\upsilon
\let\be=\beta

\let\De=\Delta \let\Ga=\Gamma \let\La=\Lambda \let\Om=\Omega
\let\Th=\Theta

% \MATHCAL - \ca ----------------------------------------------------


\DeclareMathAlphabet{\mathpzc}{OT1}{pzc}{m}{it}


% ABBREVIATION ------------------------------------------------------

\def\bea{\begin{eqnarray}}
	\def\eea{\end{eqnarray}}
\def\ba{\begin{array}}
	\def\ea{\end{array}}
\def\n{\nonumber}
\def\c{\mathscr}
\def\la{\langle}
\def\ra{\rangle}


\begin{document}
	\title{Resetting photons}	
	\author{Guilherme Eduardo Freire Oliveira, Christian Maes and Kasper Meerts\\ {\it Instituut voor Theoretische Fysica, KU Leuven}}

\begin{abstract}
Starting from a frequency diffusion process for a tagged photon which simulates relaxation to the Planck law,  we introduce a resetting where photons lower their frequency at random times.
We consider two versions, one where the resetting to low frequency is independent of the existing frequency and a second case where the reduction in frequency scales with the original frequency.  The result is a nonlinear Markov process where the stationary distribution modifies the Planck law by abundanace of low frequency occupation. The physical relevance of such photon resetting processes can be found in explorations of nonequilibrium effects, e.g. via random expansions of a confined plasma or photon gas. 
\end{abstract}
\maketitle

\tableofcontents
\section{Introduction}
Resetting has been introduced and added to diffusion processes for a variety of reasons since its original conception, \cite{evans}.  Typically, optimizing search strategies have been the underlying motivation, but one can also imagine physical resettings.  By physical resettings we mean the result of a time-dependent potential, for which there are random moments of confinement, or the random appearance in certain locatons of attractors, or the random contraction of an enclosed volume.  In the present paper, we investigate a new scenario where resetting is applied in frequency space of photons.  In that way we also explore physically motivated nonequilibrium effects on the Planck distribution.\\

Resetting photons to a lower frequency refers to reducing the wave vector (in the reciprocal lattice), which, in real space, refers to an expansion.  The physical mechanism we imagine here is that of a confined plasma where at random moments the confinement is lifted....

\section{The Kompaneets process}

\subsection{Kompaneets equation}
As reference process we consider a fluctuation dynamics which realizes the Kompaneets equation as its nonlinear Fokker-Planck equation.  Here the physical realization is
relaxation towards equilibrium of a photon gas in contact with a nondegenerate, nonrelativistic electron bath in thermal equilibrium at temperature $T$.  We concentrate here on Compton scattering as main mechanism  In 1957 Kompaneets \cite{kompa}  stared from a semi-classical Boltzmann equation to arrive at an equation for $n(t,\omega)$, the occupation number distribution function  at frequency $\omega$ of the photon gas at time $t$:
\begin{equation}\label{ke}
\omega^2\frac{\partial n}{\partial t}(t,\omega)= \frac{n_e\sigma_T 
	c}{m_e c^2}\frac{\partial }{\partial \omega}\omega^4\left\{k_B T 
\frac{\partial n}{\partial \omega}(t,\omega) + 
\hbar\left[1+n(t,\omega)\right]n(t,\omega)\right\}
\end{equation}
The constant $\sigma_T$ is the Thomson total cross section, and $n_e,m_e$ are  the density and mass of the electrons, respectively.
Note the stimulated emission (induced Compton scattering, \cite{liedahl, blandford})  in the  nonlinearity (in the second term) of \eqref{ke}.  Stationarity is achieved when $n(t,\omega)$ follows the Bose-Einstein distribution.  

The Kompaneets equation remains essential for the dynamical understanding of the cosmic microwave backgrond and related Sunyaev-Zeldovich effect \cite{sunyaeveffect,sunyaev}.
We refer to excellent reviews \cite{practical,gui,zeldovich} for more detaile. We skip here the many possible extensions and generalizations; see e.g. \cite{buet, pitrou,barbosa, brown, itoh, itoh2, cooper, kohyama1, kohyama2, kohyama3,paper}.\\

A dimensionless Kompaneets equation  employs an average photon occupation number $n(t, x)$ at dimensionless frequency $x= \hbar \omega/k_B T$,
\begin{equation}\label{ake}
x^2\frac{\partial n}{\partial y}(y,x) = \frac{\partial }{\partial x}x^4\left\{
\frac{\partial n}{\partial x}(y,x) + 
\left[1+n(y,x)\right]n(y,x)\right\}
\end{equation}
with dimensionless Compton optical depth
\[y = \frac{ k_B T }{m_e c^2} n_e \sigma_T c \, t\coloneqq \frac{ t}{\tau_C}\]
where $\tau_C$ is  the  characteristic time in which photons update their frequency as the result of Compton scattering with thermal electrons.  Note here that the scattering can be interpreted as a Doppler shift
\begin{equation}\label{shift}
\left\langle\frac{1}{2\tau}\left(\frac{\Delta\omega}{\omega}\right)^2\right\rangle\approx \frac{ k_B T }{m_e c^2} n_e \sigma_T c= \frac{1}{\tau_C}
\end{equation} 
for $\tau = \ell/c$ ,  the average collision rate in terms of the mean free path of photons $\ell=(n_e\sigma_T)^{-1}$.\\

The Kompaneets equation \eqref{ke} can be rewritten still in terms of the photon density.  Here we assume confinement of photons in a box of volume $V$ with periodic boundary conditions.  The density of states is
\[
g(\vb{k}) \dd^3 \vb{k} = \frac{2V}{(2\pi)^3} \dd^3 \vb{k} = \frac{2V}{(2\pi)^3} 
4\pi k^2 \dd k
\]
where $\vb{k}$ is the wave vector.  Under isotropy, which we will always assume, and for $x = \hbar \omega / k_B T$,
\[
g(x) \dd x = \frac{2V}{(2\pi)^3} \left( \frac{k_B T}{\hbar c} \right)^3 4\pi 
x^2 \dd x
\] 
It allows to define the spectral number density, i.e., the number of photons with energy between $E$ and $E+\dd E$ as 
\begin{equation}\label{eq:snd}
u(y,x) = g(x) n(y,x) = V \frac{1}{\pi^2} \left( \frac{k_B T}{\hbar c} \right)^3 
x^2 n(y,x)
\end{equation}
The total number of photons is then
\[ N = \int_0^\infty \dd{x} u(y,x)
 \]
and the spectral probability density equals
\begin{equation}\label{eq:spd}
\rho(y,x) = \frac{V}{N} \frac{1}{\pi^2} \left(\frac{k_B T}{\hbar c}\right)^3 
x^2 n(y,x) = \frac{x^2 n(y,x)}{2\zeta(3) Z}
\end{equation}
We put $\zeta(3) = 1/2 \int_0^\infty \id x x^2/(e^x-1) \simeq 1.202$
so that $Z$ depends on the temperature and is proportional 
to the number of photons per volume, and serves to normalize $\rho$,  It can be interpreted as the ratio of the photon density to that 
of the Planck distribution corresponding to the same temperature.
 When $n(y,x)= n_{BE}(x) = 1 / (\exp(x) - 1)$ (Bose-Einstein distribution) the photon number 
equals
\[
N_{BE} = 2 \zeta(3) \frac{V}{\pi^2} \left( \frac{k_B T}{\hbar c} \right)^3
\]
corresponding to a spectral probability density with $Z = 1, \rho_{BE}(x) = \frac{x^2 n_{BE}(x)}{2\zeta(3)}$.\\
Then,  in terms of the photon spectral density \eqref{eq:spd}, the Kompaneets equation \eqref{ake} becomes
\begin{equation}\label{kp}
\frac{\partial \rho}{\partial y} (y,x) = -\frac{\partial}{\partial x}\left[\left(4x- x^2\left(1+2\zeta(3) Z\,\frac{\rho(y,x)}{x^2}\right)\right)\rho(y,x)\right] + \frac{\partial^2}{\partial x^2}\left[x^2 \rho(y,x)\right]
\end{equation}
Note that for 
$Z\rightarrow 0$, the Wien distribution $\rho_\text{Wien}(x) = x^2 e^{-x}/2$ becomes stationary, which gives less weight to low frequency compared to the Planck law.

\subsection{Stochastic process}
Tthe Kompaneets equation is \textbf{positivity} preserving \cite{positivity}. It allows therefore a probabilistic interpretation as nonlinear Fokker-Planck equation.  That was explicitly realized in \cite{fre}.\\

We constructed there the tagged photon diffusion process,
\begin{equation} \label{kp-ito2}
\dot x	= \frac{\dd {B}}{\dd x}(x) - \beta {B}(x) {U'}(x)(1+n(x,t)) +   2\frac{{B}(x)}{x} + \sqrt{2{B}(x)}\, \xi_t
\end{equation}
where $\xi_t$ is standard white noise.  That It\^o stochastic process \eqref{kp-ito2} is our Kompaneets process, a (nonlinear)Langevin dynamics associated to the Kompaneets equation \eqref{kp} in   the case where
\[
U(x) = x,\qquad B(x) = x^2
\]  
One can interpret it as a mean-field Markov process for the tagged photon, where the field $n(t,x)$  represents the empirical occupations, followinf the ideas of e.g,  the McKean-Vlasov equation \cite{VM}.  The specific mult-photon diffusion limit was discussed in \cite{freq}.\\
  We remark that the white noise can lead to (unphysical) negative values of $x$.  The {\it ad hoc} remedy is to puch by hand (in the simulation) the frequecny value positive, effectively using a boundary condition at $x=0$.   Physically speaking, reaction processes such as Bremsstrahlung and double Compton scattering control the photon number density but we ignore here the detailed implementation of these processes.\\
  
  As an illustration of the soundness of the process... we reproduce here....
  
\begin{figure}
	\includegraphics{{spd_N-1e6.00_dx-0.05_t-3_dt-0.0001_ic-hotplanck}.pdf}
	\caption{Result of Kompaneets, for three units of time.  Needs indication of what in axes.  Mention maximum excess of 1/1000.  Mention very fastg prethermalization, slow final thermalization --- low frequency going like $x$.}
	\label{fig:spd-kompaneets}
\end{figure}

\section{Resetting Kompaneets process}

\subsection{Process}
Physical motivation plus definiton of two versions of resettings.

We will consider the same dynamics as before, adding Poissonian resetting with 
a constant rate. We introduce the resetting rate $r$. Three methods are 
introduced for resetting, the so-called "division" method, where the energy of 
the photon is divided by a constant factor $d$, the "uniform" method, where the 
photon's energy is reset to a random value uniformly distributed between 0 and 
a cutoff $x_0$, and finally the "exponential" method, where the new energy of 
the photon follows an exponential distribution with scale $x_0$.

For a justification of the mechanisms underlying this resetting we turn our 
attention towards two phenomena: the metric expansion of space and the Doppler 
effect. From the beginning of the lepton era to recombination space has 
expanded millionfold, and from recombination until now the scale factor has 
increased by another factor of one thousand. In the highly symmetric FLRW 
solution of the Einstein field equations, this scale factor has risen 
continuously and homogeneously. Moving away from this highly idealized 
solution, we can imagine more localized, abrupt increases, applying only to a 
fraction of photons.  As a photon's frequency is inversely proportional to the 
scale factor, this would in effect be a stochastic division of photon 
frequencies by some large factor. This process is reminiscent of the shift in 
frequency due to the Doppler, and in fact mathematically and conceptually this 
process is identical to that of Doppler shifts from expanding matter. In this 
regard, one can also consider a cavity with randomly, quickly receding walls.  
Any photon catching up to these walls would again undergo a downwards shift in 
frequency.

\textbf{Kasper:} Is there a point in having multiple methods? Looking at figure 
\ref{fig:comparison}, with the right tuning of parameters all three methods of 
resetting give identical results. The choices of the scale, interval and 
divisor are such that the expected value of a photon after resetting is the 
same, $0.05$ in this case.

This implies that the position of the photon is updated by the following 
stochastic rules, in an infinitesimal interval $\dd t$ we have
\begin{align}\label{eq:resetted-ito}
x(t + \dd t) &= x(t) + D(x,t,n_t) \dd t + \sqrt{2{B}(x)}\, \xi_t \sqrt{\dd 
t}&&\text{with probability $(1-r)\dd t$}\\\nonumber
&= \begin{cases}x(t) / d &\text{"division" method}\\
x\sim \operatorname{Uniform}(0,x_0)&\text{"uniform" method}\\
x\sim \operatorname{Exp}(x_0)&\text{"exponential" method}\\ 
\end{cases}&&\text{with probability $r \dd t$}
\end{align}

To simulate these dynamics, we revisit the procedure from (cite vorige paper?).  
Adding to this, at every timestep, for every particle, we perform the resetting 
step with a probability $r \Delta t$. The timestep $\Delta t$ is chosen such 
that this product is much smaller than 1, making sure that the probability of 
two resets happening in the same timestep remains negligible.

One particular difficulty here lies in the fact that these dynamics are quite 
prone to the formation of condensates. In particular, the Planck distribution 
at zero chemical potential is on the knife-edge of this phenomenon, if there is 
any excess in particles, a non-zero probability flux will come into being at 
the origin, leading to a Dirac delta mass at $x=0$. More precisely, letting $f 
= x^2 n$, the Kompaneets equation takes the form
\begin{align*}
\dot{f} = -\partial_x j &&\text{with}&&j = 2xf -x^2 f'-x^2 f-f^2.
\end{align*}
If we substitute $f=\frac{x^2}{e^x - 1} + \delta$ we find that $j = \delta^2 + 
O(x)$, hence no probability can be allowed near $x=0$ to first order. To remedy 
this, we assume the photon gas is slightly rarefied, having a particle density 
that is 90\% that of a photon gas following the Planck distribution at the same 
temperature. This is also the distribution with which we compare the results of 
the simulation
\begin{align}
\rho(x) = \frac{1}{0.9\cdot2\zeta(3)}\frac{1}{e^{x+\mu}-1}&&\text{with $\mu 
\approx 0.08$}
\end{align}

\begin{figure}
\begin{subfigure}{\textwidth}
\includegraphics[width=0.48\textwidth]{{spd_N-1e5.95_t-10_Z-0.9000_reset_rate-0.1000_divisor-10.00}.pdf}\hfill%
\includegraphics[width=0.48\textwidth]{{excess_N-1e5.95_t-10_Z-0.9000_reset_rate-0.1000_divisor-10.00}.pdf}
\caption{$r=\num{e-1}$, $d=10$, simulation stopped}
\end{subfigure}
\begin{subfigure}{\textwidth}
\includegraphics[width=0.48\textwidth]{{spd_N-1e5.95_t-10_Z-0.9000_reset_rate-0.0100_divisor-10.00}.pdf}\hfill%
\includegraphics[width=0.48\textwidth]{{excess_N-1e5.95_t-10_Z-0.9000_reset_rate-0.0100_divisor-10.00}.pdf}
\caption{$r=\num{e-2}$, $d=10$}
\end{subfigure}
\begin{subfigure}{\textwidth}
\includegraphics[width=0.48\textwidth]{{spd_N-1e5.95_t-10_Z-0.9000_reset_rate-0.0100_divisor-100.00}.pdf}\hfill%
\includegraphics[width=0.48\textwidth]{{excess_N-1e5.95_t-10_Z-0.9000_reset_rate-0.0100_divisor-100.00}.pdf}
\caption{$r=\num{e-2}$, $d=100$, simulation stopped}
\end{subfigure}
\end{figure}

\begin{figure}
\begin{subfigure}{\textwidth}
\includegraphics[width=0.48\textwidth]{{spd_N-1e5.95_t-10_Z-0.9000_reset_rate-0.0100_divisor-1000.00}.pdf}\hfill%
\includegraphics[width=0.48\textwidth]{{excess_N-1e5.95_t-10_Z-0.9000_reset_rate-0.0100_divisor-1000.00}.pdf}
\caption{$r=\num{e-2}$, $d=1000$, simulation stopped}
\end{subfigure}
\begin{subfigure}{\textwidth}
\includegraphics[width=0.48\textwidth]{{spd_N-1e5.95_t-10_Z-0.9000_reset_rate-0.0010_divisor-1000.00}.pdf}\hfill%
\includegraphics[width=0.48\textwidth]{{excess_N-1e5.95_t-10_Z-0.9000_reset_rate-0.0010_divisor-1000.00}.pdf}
\caption{$r=\num{e-3}$, $d=1000$}
\end{subfigure}
\begin{subfigure}{\textwidth}
\includegraphics[width=0.48\textwidth]{{spd_N-1e5.95_t-10_Z-0.9000_reset_rate-0.0001_divisor-1000.00}.pdf}\hfill%
\includegraphics[width=0.48\textwidth]{{excess_N-1e5.95_t-10_Z-0.9000_reset_rate-0.0001_divisor-1000.00}.pdf}
\caption{$r=\num{e-4}$, $d=1000$}
\end{subfigure}
\end{figure}

\begin{figure}
\begin{subfigure}{\textwidth}
\includegraphics[width=0.48\textwidth]{{spd_N-1e5.95_t-10_Z-0.9000_reset_rate-0.0100_interval-0.10}.pdf}\hfill%
\includegraphics[width=0.48\textwidth]{{excess_N-1e5.95_t-10_Z-0.9000_reset_rate-0.0100_interval-0.10}.pdf}
\caption{$r=\num{e-2}$, $x_0=0.1$, simulation stopped}
\end{subfigure}
\begin{subfigure}{\textwidth}
\includegraphics[width=0.48\textwidth]{{spd_N-1e5.95_t-10_Z-0.9000_reset_rate-0.0010_interval-0.10}.pdf}\hfill%
\includegraphics[width=0.48\textwidth]{{excess_N-1e5.95_t-10_Z-0.9000_reset_rate-0.0010_interval-0.10}.pdf}
\caption{$r=\num{e-3}$, $x_0=0.1$}
\end{subfigure}
\begin{subfigure}{\textwidth}
\includegraphics[width=0.48\textwidth]{{spd_N-1e5.95_t-10_Z-0.9000_reset_rate-0.0001_interval-0.10}.pdf}\hfill%
\includegraphics[width=0.48\textwidth]{{excess_N-1e5.95_t-10_Z-0.9000_reset_rate-0.0001_interval-0.10}.pdf}
\caption{$r=\num{e-4}$, $x_0=0.1$}
\end{subfigure}
\end{figure}

\begin{figure}
\begin{subfigure}{\textwidth}
\includegraphics[width=0.48\textwidth]{{spd_N-1e6.95_t-10_Z-0.9000_reset_rate-0.0010_interval-0.10}.pdf}\hfill%
\includegraphics[width=0.48\textwidth]{{excess_N-1e6.95_t-10_Z-0.9000_reset_rate-0.0010_interval-0.10}.pdf}
\caption{$r=\num{e-3}$, "interval", $x_0=0.1$}
\end{subfigure}
\begin{subfigure}{\textwidth}
\includegraphics[width=0.48\textwidth]{{spd_N-1e6.95_t-10_Z-0.9000_reset_rate-0.0010_scale-0.05}.pdf}\hfill%
\includegraphics[width=0.48\textwidth]{{excess_N-1e6.95_t-10_Z-0.9000_reset_rate-0.0010_scale-0.05}.pdf}
\caption{$r=\num{e-3}$, "exponential", $x_0=0.05$}
\end{subfigure}
\begin{subfigure}{\textwidth}
\includegraphics[width=0.48\textwidth]{{spd_N-1e6.95_t-10_Z-0.9000_reset_rate-0.0010_divisor-60.00}.pdf}\hfill%
\includegraphics[width=0.48\textwidth]{{excess_N-1e6.95_t-10_Z-0.9000_reset_rate-0.0010_divisor-60.00}.pdf}
\caption{$r=\num{e-4}$, $d=60$}
\end{subfigure}
\caption{Comparison between methods}\label{fig:comparison}
\end{figure}

\subsection{Discussion of results}




\section {Conclusions}\label{con}





\bibliographystyle{abbrv}
\bibliography{langevin-kompaneets}

\end{document}
